% Options for packages loaded elsewhere
\PassOptionsToPackage{unicode}{hyperref}
\PassOptionsToPackage{hyphens}{url}
%
\documentclass[
]{article}
\usepackage{lmodern}
\usepackage{amssymb,amsmath}
\usepackage{ifxetex,ifluatex}
\ifnum 0\ifxetex 1\fi\ifluatex 1\fi=0 % if pdftex
  \usepackage[T1]{fontenc}
  \usepackage[utf8]{inputenc}
  \usepackage{textcomp} % provide euro and other symbols
\else % if luatex or xetex
  \usepackage{unicode-math}
  \defaultfontfeatures{Scale=MatchLowercase}
  \defaultfontfeatures[\rmfamily]{Ligatures=TeX,Scale=1}
\fi
% Use upquote if available, for straight quotes in verbatim environments
\IfFileExists{upquote.sty}{\usepackage{upquote}}{}
\IfFileExists{microtype.sty}{% use microtype if available
  \usepackage[]{microtype}
  \UseMicrotypeSet[protrusion]{basicmath} % disable protrusion for tt fonts
}{}
\makeatletter
\@ifundefined{KOMAClassName}{% if non-KOMA class
  \IfFileExists{parskip.sty}{%
    \usepackage{parskip}
  }{% else
    \setlength{\parindent}{0pt}
    \setlength{\parskip}{6pt plus 2pt minus 1pt}}
}{% if KOMA class
  \KOMAoptions{parskip=half}}
\makeatother
\usepackage{xcolor}
\IfFileExists{xurl.sty}{\usepackage{xurl}}{} % add URL line breaks if available
\IfFileExists{bookmark.sty}{\usepackage{bookmark}}{\usepackage{hyperref}}
\hypersetup{
  pdftitle={IV Example},
  hidelinks,
  pdfcreator={LaTeX via pandoc}}
\urlstyle{same} % disable monospaced font for URLs
\usepackage[margin=1in]{geometry}
\usepackage{color}
\usepackage{fancyvrb}
\newcommand{\VerbBar}{|}
\newcommand{\VERB}{\Verb[commandchars=\\\{\}]}
\DefineVerbatimEnvironment{Highlighting}{Verbatim}{commandchars=\\\{\}}
% Add ',fontsize=\small' for more characters per line
\usepackage{framed}
\definecolor{shadecolor}{RGB}{248,248,248}
\newenvironment{Shaded}{\begin{snugshade}}{\end{snugshade}}
\newcommand{\AlertTok}[1]{\textcolor[rgb]{0.94,0.16,0.16}{#1}}
\newcommand{\AnnotationTok}[1]{\textcolor[rgb]{0.56,0.35,0.01}{\textbf{\textit{#1}}}}
\newcommand{\AttributeTok}[1]{\textcolor[rgb]{0.77,0.63,0.00}{#1}}
\newcommand{\BaseNTok}[1]{\textcolor[rgb]{0.00,0.00,0.81}{#1}}
\newcommand{\BuiltInTok}[1]{#1}
\newcommand{\CharTok}[1]{\textcolor[rgb]{0.31,0.60,0.02}{#1}}
\newcommand{\CommentTok}[1]{\textcolor[rgb]{0.56,0.35,0.01}{\textit{#1}}}
\newcommand{\CommentVarTok}[1]{\textcolor[rgb]{0.56,0.35,0.01}{\textbf{\textit{#1}}}}
\newcommand{\ConstantTok}[1]{\textcolor[rgb]{0.00,0.00,0.00}{#1}}
\newcommand{\ControlFlowTok}[1]{\textcolor[rgb]{0.13,0.29,0.53}{\textbf{#1}}}
\newcommand{\DataTypeTok}[1]{\textcolor[rgb]{0.13,0.29,0.53}{#1}}
\newcommand{\DecValTok}[1]{\textcolor[rgb]{0.00,0.00,0.81}{#1}}
\newcommand{\DocumentationTok}[1]{\textcolor[rgb]{0.56,0.35,0.01}{\textbf{\textit{#1}}}}
\newcommand{\ErrorTok}[1]{\textcolor[rgb]{0.64,0.00,0.00}{\textbf{#1}}}
\newcommand{\ExtensionTok}[1]{#1}
\newcommand{\FloatTok}[1]{\textcolor[rgb]{0.00,0.00,0.81}{#1}}
\newcommand{\FunctionTok}[1]{\textcolor[rgb]{0.00,0.00,0.00}{#1}}
\newcommand{\ImportTok}[1]{#1}
\newcommand{\InformationTok}[1]{\textcolor[rgb]{0.56,0.35,0.01}{\textbf{\textit{#1}}}}
\newcommand{\KeywordTok}[1]{\textcolor[rgb]{0.13,0.29,0.53}{\textbf{#1}}}
\newcommand{\NormalTok}[1]{#1}
\newcommand{\OperatorTok}[1]{\textcolor[rgb]{0.81,0.36,0.00}{\textbf{#1}}}
\newcommand{\OtherTok}[1]{\textcolor[rgb]{0.56,0.35,0.01}{#1}}
\newcommand{\PreprocessorTok}[1]{\textcolor[rgb]{0.56,0.35,0.01}{\textit{#1}}}
\newcommand{\RegionMarkerTok}[1]{#1}
\newcommand{\SpecialCharTok}[1]{\textcolor[rgb]{0.00,0.00,0.00}{#1}}
\newcommand{\SpecialStringTok}[1]{\textcolor[rgb]{0.31,0.60,0.02}{#1}}
\newcommand{\StringTok}[1]{\textcolor[rgb]{0.31,0.60,0.02}{#1}}
\newcommand{\VariableTok}[1]{\textcolor[rgb]{0.00,0.00,0.00}{#1}}
\newcommand{\VerbatimStringTok}[1]{\textcolor[rgb]{0.31,0.60,0.02}{#1}}
\newcommand{\WarningTok}[1]{\textcolor[rgb]{0.56,0.35,0.01}{\textbf{\textit{#1}}}}
\usepackage{longtable,booktabs}
% Correct order of tables after \paragraph or \subparagraph
\usepackage{etoolbox}
\makeatletter
\patchcmd\longtable{\par}{\if@noskipsec\mbox{}\fi\par}{}{}
\makeatother
% Allow footnotes in longtable head/foot
\IfFileExists{footnotehyper.sty}{\usepackage{footnotehyper}}{\usepackage{footnote}}
\makesavenoteenv{longtable}
\usepackage{graphicx,grffile}
\makeatletter
\def\maxwidth{\ifdim\Gin@nat@width>\linewidth\linewidth\else\Gin@nat@width\fi}
\def\maxheight{\ifdim\Gin@nat@height>\textheight\textheight\else\Gin@nat@height\fi}
\makeatother
% Scale images if necessary, so that they will not overflow the page
% margins by default, and it is still possible to overwrite the defaults
% using explicit options in \includegraphics[width, height, ...]{}
\setkeys{Gin}{width=\maxwidth,height=\maxheight,keepaspectratio}
% Set default figure placement to htbp
\makeatletter
\def\fps@figure{htbp}
\makeatother
\setlength{\emergencystretch}{3em} % prevent overfull lines
\providecommand{\tightlist}{%
  \setlength{\itemsep}{0pt}\setlength{\parskip}{0pt}}
\setcounter{secnumdepth}{-\maxdimen} % remove section numbering

\title{IV Example}
\author{}
\date{\vspace{-2.5em}}

\begin{document}
\maketitle

来源: \url{https://bookdown.org/ccolonescu/RPoE4/random-regressors.html}

\begin{Shaded}
\begin{Highlighting}[]
\KeywordTok{library}\NormalTok{(AER) }\CommentTok{#for `ivreg()`}
\KeywordTok{library}\NormalTok{(lmtest) }\CommentTok{#for `coeftest()` and `bptest()`.}
\KeywordTok{library}\NormalTok{(broom) }\CommentTok{#for `glance(`) and `tidy()`}
\KeywordTok{library}\NormalTok{(PoEdata) }\CommentTok{#for PoE4 datasets}
\KeywordTok{library}\NormalTok{(car) }\CommentTok{#for `hccm()` robust standard errors}
\KeywordTok{library}\NormalTok{(sandwich)}
\KeywordTok{library}\NormalTok{(knitr) }\CommentTok{#for making neat tables with `kable()`}
\KeywordTok{library}\NormalTok{(stargazer) }
\KeywordTok{library}\NormalTok{(sjPlot)}
\KeywordTok{library}\NormalTok{(sjmisc)}
\KeywordTok{library}\NormalTok{(plm)}
\KeywordTok{library}\NormalTok{(sjlabelled)}
\end{Highlighting}
\end{Shaded}

\hypertarget{ux6a21ux578b}{%
\section{模型}\label{ux6a21ux578b}}

\[\begin{equation}
log(wage)=\beta_{1}+\beta_{2}educ+\beta_{3}exper+\beta_{4}exper^2+e
\label{eq:wagelm10}
\end{equation}\]

\hypertarget{ux6570ux636eux4e0eux7b2cux4e00ux9636ux6bb5ux56deux5f52}{%
\section{数据与第一阶段回归}\label{ux6570ux636eux4e0eux7b2cux4e00ux9636ux6bb5ux56deux5f52}}

\[\begin{equation}
educ=\gamma_{1}+\gamma_{2}exper+\gamma_{3}exper^2+\theta_{1}mothereduc+\nu_{educ}
\label{eq:firstStageEduc10}
\end{equation}\]

\begin{Shaded}
\begin{Highlighting}[]
\KeywordTok{data}\NormalTok{(}\StringTok{"mroz"}\NormalTok{, }\DataTypeTok{package=}\StringTok{"PoEdata"}\NormalTok{)}
\NormalTok{mroz1 <-}\StringTok{ }\NormalTok{mroz[mroz}\OperatorTok{$}\NormalTok{lfp}\OperatorTok{==}\DecValTok{1}\NormalTok{,] }\CommentTok{#restricts sample to lfp=1}
\NormalTok{educ.ols <-}\StringTok{ }\KeywordTok{lm}\NormalTok{(educ}\OperatorTok{~}\NormalTok{exper}\OperatorTok{+}\KeywordTok{I}\NormalTok{(exper}\OperatorTok{^}\DecValTok{2}\NormalTok{)}\OperatorTok{+}\NormalTok{mothereduc, }\DataTypeTok{data=}\NormalTok{mroz1)}
\KeywordTok{kable}\NormalTok{(}\KeywordTok{tidy}\NormalTok{(educ.ols), }\DataTypeTok{digits=}\DecValTok{4}\NormalTok{, }\DataTypeTok{align=}\StringTok{'c'}\NormalTok{,}\DataTypeTok{caption=}
  \StringTok{"First stage in the 2SLS model for the 'wage' equation"}\NormalTok{)}
\end{Highlighting}
\end{Shaded}

\begin{longtable}[]{@{}ccccc@{}}
\caption{First stage in the 2SLS model for the `wage'
equation}\tabularnewline
\toprule
term & estimate & std.error & statistic & p.value\tabularnewline
\midrule
\endfirsthead
\toprule
term & estimate & std.error & statistic & p.value\tabularnewline
\midrule
\endhead
(Intercept) & 9.7751 & 0.4239 & 23.0605 & 0.0000\tabularnewline
exper & 0.0489 & 0.0417 & 1.1726 & 0.2416\tabularnewline
I(exper\^{}2) & -0.0013 & 0.0012 & -1.0290 & 0.3040\tabularnewline
mothereduc & 0.2677 & 0.0311 & 8.5992 & 0.0000\tabularnewline
\bottomrule
\end{longtable}

\hypertarget{ux7b2cux4e8cux9636ux6bb5ux56deux5f52}{%
\section{第二阶段回归}\label{ux7b2cux4e8cux9636ux6bb5ux56deux5f52}}

\begin{Shaded}
\begin{Highlighting}[]
\NormalTok{educHat <-}\StringTok{ }\KeywordTok{fitted}\NormalTok{(educ.ols)}
\NormalTok{wage}\FloatTok{.2}\NormalTok{sls <-}\StringTok{ }\KeywordTok{lm}\NormalTok{(}\KeywordTok{log}\NormalTok{(wage)}\OperatorTok{~}\NormalTok{educHat}\OperatorTok{+}\NormalTok{exper}\OperatorTok{+}\KeywordTok{I}\NormalTok{(exper}\OperatorTok{^}\DecValTok{2}\NormalTok{), }\DataTypeTok{data=}\NormalTok{mroz1)}
\KeywordTok{kable}\NormalTok{(}\KeywordTok{tidy}\NormalTok{(wage}\FloatTok{.2}\NormalTok{sls), }\DataTypeTok{digits=}\DecValTok{4}\NormalTok{, }\DataTypeTok{align=}\StringTok{'c'}\NormalTok{,}\DataTypeTok{caption=}
  \StringTok{"Second stage in the 2SLS model for the 'wage' equation"}\NormalTok{)}
\end{Highlighting}
\end{Shaded}

\begin{longtable}[]{@{}ccccc@{}}
\caption{Second stage in the 2SLS model for the `wage'
equation}\tabularnewline
\toprule
term & estimate & std.error & statistic & p.value\tabularnewline
\midrule
\endfirsthead
\toprule
term & estimate & std.error & statistic & p.value\tabularnewline
\midrule
\endhead
(Intercept) & 0.1982 & 0.4933 & 0.4017 & 0.6881\tabularnewline
educHat & 0.0493 & 0.0391 & 1.2613 & 0.2079\tabularnewline
exper & 0.0449 & 0.0142 & 3.1668 & 0.0017\tabularnewline
I(exper\^{}2) & -0.0009 & 0.0004 & -2.1749 & 0.0302\tabularnewline
\bottomrule
\end{longtable}

\hypertarget{ux4feeux6b63ux6807ux51c6ux8befux4f30ux8ba1}{%
\section{修正标准误估计}\label{ux4feeux6b63ux6807ux51c6ux8befux4f30ux8ba1}}

\begin{Shaded}
\begin{Highlighting}[]
\KeywordTok{data}\NormalTok{(}\StringTok{"mroz"}\NormalTok{, }\DataTypeTok{package=}\StringTok{"PoEdata"}\NormalTok{)}
\NormalTok{mroz1 <-}\StringTok{ }\NormalTok{mroz[mroz}\OperatorTok{$}\NormalTok{lfp}\OperatorTok{==}\DecValTok{1}\NormalTok{,] }\CommentTok{#restricts sample to lfp=1.}
\NormalTok{mroz1.ols <-}\StringTok{ }\KeywordTok{lm}\NormalTok{(}\KeywordTok{log}\NormalTok{(wage)}\OperatorTok{~}\NormalTok{educ}\OperatorTok{+}\NormalTok{exper}\OperatorTok{+}\KeywordTok{I}\NormalTok{(exper}\OperatorTok{^}\DecValTok{2}\NormalTok{), }\DataTypeTok{data=}\NormalTok{mroz1)}
\NormalTok{mroz1.iv <-}\StringTok{ }\KeywordTok{ivreg}\NormalTok{(}\KeywordTok{log}\NormalTok{(wage)}\OperatorTok{~}\NormalTok{educ}\OperatorTok{+}\NormalTok{exper}\OperatorTok{+}\KeywordTok{I}\NormalTok{(exper}\OperatorTok{^}\DecValTok{2}\NormalTok{)}\OperatorTok{|}
\StringTok{            }\NormalTok{exper}\OperatorTok{+}\KeywordTok{I}\NormalTok{(exper}\OperatorTok{^}\DecValTok{2}\NormalTok{)}\OperatorTok{+}\NormalTok{mothereduc, }\DataTypeTok{data=}\NormalTok{mroz1)}
\NormalTok{mroz1.iv1 <-}\StringTok{ }\KeywordTok{ivreg}\NormalTok{(}\KeywordTok{log}\NormalTok{(wage)}\OperatorTok{~}\NormalTok{educ}\OperatorTok{+}\NormalTok{exper}\OperatorTok{+}\KeywordTok{I}\NormalTok{(exper}\OperatorTok{^}\DecValTok{2}\NormalTok{)}\OperatorTok{|}
\StringTok{            }\NormalTok{exper}\OperatorTok{+}\KeywordTok{I}\NormalTok{(exper}\OperatorTok{^}\DecValTok{2}\NormalTok{)}\OperatorTok{+}\NormalTok{mothereduc}\OperatorTok{+}\NormalTok{fathereduc,}
            \DataTypeTok{data=}\NormalTok{mroz1)}
\KeywordTok{tab_model}\NormalTok{(mroz1.ols, wage}\FloatTok{.2}\NormalTok{sls, mroz1.iv, mroz1.iv1,}
          \DataTypeTok{emph.p =}\NormalTok{ T, }\DataTypeTok{robust =}\NormalTok{ T,}
          \DataTypeTok{show.ci =}\NormalTok{ F, }\DataTypeTok{collapse.se =}\NormalTok{ T, }\DataTypeTok{string.pred =} \StringTok{"Coeffcient"}\NormalTok{, }\DataTypeTok{show.se =}\NormalTok{ F, }
          \DataTypeTok{dv.labels =} \KeywordTok{c}\NormalTok{(}\StringTok{"OLS"}\NormalTok{, }\StringTok{"explicit 2SLS"}\NormalTok{, }\StringTok{"IV1"}\NormalTok{, }\StringTok{"IV2"}\NormalTok{),}
          \DataTypeTok{p.style =} \StringTok{"a"}\NormalTok{)}
\end{Highlighting}
\end{Shaded}

~

OLS

explicit 2SLS

IV1

IV2

Coeffcient

Estimates

Estimates

Estimates

Estimates

(Intercept)

-0.52 *(0.20)

0.20 (0.51)

0.20 (0.49)

0.05 (0.43)

educ

0.11 ***(0.01)

0.05 (0.04)

0.06 (0.03)

exper

0.04 **(0.02)

0.04 **(0.02)

0.04 **(0.02)

0.04 **(0.02)

exper\^{}2

-0.00 (0.00)

-0.00 *(0.00)

-0.00 *(0.00)

-0.00 *(0.00)

educHat

0.05 (0.04)

Observations

428

428

428

428

R2 / R2 adjusted

0.157 / 0.151

0.046 / 0.039

0.123 / 0.117

0.136 / 0.130

\begin{itemize}
\tightlist
\item
  p\textless0.05~~~** p\textless0.01~~~*** p\textless0.001
\end{itemize}

\hypertarget{ux5f31ux5de5ux5177ux53d8ux91cfux68c0ux9a8c}{%
\section{弱工具变量检验}\label{ux5f31ux5de5ux5177ux53d8ux91cfux68c0ux9a8c}}

\[\begin{equation}
educ=\gamma_{1}+\gamma_{2}exper+\gamma_{3} exper^2+\theta_{1} mothereduc+ \theta_{2}fathereduc+\nu
\label{eq:educlm10}
\end{equation}\]

\begin{Shaded}
\begin{Highlighting}[]
\NormalTok{educ.ols <-}\StringTok{ }\KeywordTok{lm}\NormalTok{(educ}\OperatorTok{~}\NormalTok{exper}\OperatorTok{+}\KeywordTok{I}\NormalTok{(exper}\OperatorTok{^}\DecValTok{2}\NormalTok{)}\OperatorTok{+}\NormalTok{mothereduc}\OperatorTok{+}\NormalTok{fathereduc, }
               \DataTypeTok{data=}\NormalTok{mroz1)}
\NormalTok{tab <-}\StringTok{ }\KeywordTok{tidy}\NormalTok{(educ.ols)}
\KeywordTok{kable}\NormalTok{(tab, }\DataTypeTok{digits=}\DecValTok{4}\NormalTok{,}
      \DataTypeTok{caption=}\StringTok{"The 'educ' first-stage equation"}\NormalTok{)}
\end{Highlighting}
\end{Shaded}

\begin{longtable}[]{@{}lrrrr@{}}
\caption{The `educ' first-stage equation}\tabularnewline
\toprule
term & estimate & std.error & statistic & p.value\tabularnewline
\midrule
\endfirsthead
\toprule
term & estimate & std.error & statistic & p.value\tabularnewline
\midrule
\endhead
(Intercept) & 9.1026 & 0.4266 & 21.3396 & 0.0000\tabularnewline
exper & 0.0452 & 0.0403 & 1.1236 & 0.2618\tabularnewline
I(exper\^{}2) & -0.0010 & 0.0012 & -0.8386 & 0.4022\tabularnewline
mothereduc & 0.1576 & 0.0359 & 4.3906 & 0.0000\tabularnewline
fathereduc & 0.1895 & 0.0338 & 5.6152 & 0.0000\tabularnewline
\bottomrule
\end{longtable}

\begin{Shaded}
\begin{Highlighting}[]
\KeywordTok{linearHypothesis}\NormalTok{(educ.ols, }\KeywordTok{c}\NormalTok{(}\StringTok{"mothereduc=0"}\NormalTok{, }\StringTok{"fathereduc=0"}\NormalTok{))}
\end{Highlighting}
\end{Shaded}

\begin{verbatim}
## Linear hypothesis test
## 
## Hypothesis:
## mothereduc = 0
## fathereduc = 0
## 
## Model 1: restricted model
## Model 2: educ ~ exper + I(exper^2) + mothereduc + fathereduc
## 
##   Res.Df    RSS Df Sum of Sq    F    Pr(>F)    
## 1    425 2219.2                                
## 2    423 1758.6  2    460.64 55.4 < 2.2e-16 ***
## ---
## Signif. codes:  0 '***' 0.001 '**' 0.01 '*' 0.05 '.' 0.1 ' ' 1
\end{verbatim}

\hypertarget{specification-tests}{%
\section{specification tests}\label{specification-tests}}

\begin{enumerate}
\def\labelenumi{\arabic{enumi}.}
\item
  内生性检验 (Hausman test for endogeneity):\(H_{0}:\;Cov(x,e)=0\)
\item
  过度识别检验:(overidentifying restrictions, or the Sargan test):
  \(H_{0}:Cov(z,e)=0\)
\end{enumerate}

\begin{Shaded}
\begin{Highlighting}[]
\KeywordTok{summary}\NormalTok{(mroz1.iv1, }\DataTypeTok{diagnostics=}\OtherTok{TRUE}\NormalTok{)}
\end{Highlighting}
\end{Shaded}

\begin{verbatim}
## 
## Call:
## ivreg(formula = log(wage) ~ educ + exper + I(exper^2) | exper + 
##     I(exper^2) + mothereduc + fathereduc, data = mroz1)
## 
## Residuals:
##     Min      1Q  Median      3Q     Max 
## -3.0986 -0.3196  0.0551  0.3689  2.3493 
## 
## Coefficients:
##               Estimate Std. Error t value Pr(>|t|)   
## (Intercept)  0.0481003  0.4003281   0.120  0.90442   
## educ         0.0613966  0.0314367   1.953  0.05147 . 
## exper        0.0441704  0.0134325   3.288  0.00109 **
## I(exper^2)  -0.0008990  0.0004017  -2.238  0.02574 * 
## 
## Diagnostic tests:
##                  df1 df2 statistic p-value    
## Weak instruments   2 423    55.400  <2e-16 ***
## Wu-Hausman         1 423     2.793  0.0954 .  
## Sargan             1  NA     0.378  0.5386    
## ---
## Signif. codes:  0 '***' 0.001 '**' 0.01 '*' 0.05 '.' 0.1 ' ' 1
## 
## Residual standard error: 0.6747 on 424 degrees of freedom
## Multiple R-Squared: 0.1357,  Adjusted R-squared: 0.1296 
## Wald test: 8.141 on 3 and 424 DF,  p-value: 2.787e-05
\end{verbatim}

检验的主要结论:

\begin{itemize}
\item
  Weak instruments test: rejects the null, meaning that at least one
  instrument is strong
\item
  (Wu-)Hausman test for endogeneity: barely rejects the null that the
  variable of concern is uncorrelated with the error term, indicating
  that educ is marginally endogenous
\item
  Sargan overidentifying restrictions: does not reject the null, meaning
  that the extra instruments are valid (are uncorrelated with the error
  term).
\end{itemize}

\end{document}
